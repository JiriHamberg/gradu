\section{Association Analysis}

Association rule mining is the task of finding associations between \textit{items} in a database of \textit{transactions}. The technique was originally developed in 1993 to identify patterns in consumers grocery purchasing behaviour. As an example, consider a database of following transactions denoting individual grocery purchases:
 
\begin{center}
    \begin{tabular}{ | l | l | }
    \hline
    \textbf{ID} & \textbf{Items} \\ \hline
    1 & bread, butter, cheese, beer \\ \hline 
    2 & diapers, beer, cookies \\ \hline 
    3 & diapers, beer, bread \\ \hline 
    4 & butter, bread, cheese \\ \hline 
    \end{tabular}
\end{center} 
 
The aim of the association analysis is then to produce a list of association rules like 

\begin{itemize}
	\item $\left\{ bread, butter \right\} \Rightarrow \left\{ cheese \right\}$	
	\item $\left\{ diapers \right\} \Rightarrow \left\{ beer \right\}$	
\end{itemize} 

A rule such as $\left\{ bread, butter \right\} \Rightarrow \left\{ cheese \right\}$	should be taken to mean "Presence of items \textit{bread} and \textit{butter} indicates the presence of item \textit{cheese}". 

\subsection{Formal Definition}


\subsection{FP-growth}