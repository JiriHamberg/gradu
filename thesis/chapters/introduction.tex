\section{Introduction}

\textit{Big Data} is a term coined to capture the essence of large datasets typical for the digital age. Big Data can be described by three aspects: \textit{volume}, \textit{velocity} and \textit{variety}, that make such data difficult to process and analyse by traditional tools and methods~\cite{doi:10.1108/LR-06-2015-0061}. Big \textit{volume} means that the data is simply too large to be processed by traditional tools. Big \textit{velocity} means that the data is growing so quickly that traditional tools cannot keep up with the pace. Big \textit{variety} can refer to high dimensionality or loosely constrained structure of a dataset.

Association rule discovery is an established data mining technique that is commonly used to discover interesting relations from a dataset without making assumptions about its structure. This thesis work explores the applicability of association rule discovery on Big Data. As a practical application of the theory, this thesis will demonstrate how the association rule discovery can be applied to discover relations between mobile device system settings and level of energy consumption from the data produced by Carat~\cite{Oliner:2013:CCE:2517351.2517354}, a collaborative energy diagnosis project.

The Carat project has collected data from over 800,000 mobile devices worldwide since its initiation in 2012~\cite{7840871}. The Carat data is collected from mobile device users that have installed the Carat mobile application to their device. An analysis server collects data samples sent by the Carat mobile applications whenever the battery level of the device changes. The data samples consist of a list of system settings, a list of currently running applications and the current battery level. The analysis server uses the collaborative measurements to identify energy consumption anomalies from the users' applications as well as to estimate the energy consumption of individual applications~\cite{Oliner:2013:CCE:2517351.2517354}.

So far the Carat dataset has not been published in its entirety to ensure the privacy of the participants. As discussed in~\cite{7840871}, it would be helpful for mobile application developers to be able to access the energy consumption data of the users of their application. As a part of this thesis work, a prototype of a Carat developer web API is implemented. The API provides a search engine for mobile application developers, allowing them to search for associations between specific system settings and energy consumption level of the mobile device.   

Key research questions that this thesis attempts to answer are:

\begin{enumerate}
	\item How can association rules be generated efficiently and scalably from Big Data? The solution should be fast enough to be used in a real-time query API. 
	
	\item How to select interesting and useful association rules from the set of all generated rules? 
	
	\item How does the discretization algorithm of continuous variables of the Carat data affect the generated association rules? 
\end{enumerate}  

%Various distributed computing technologies have emerged to tackle the difficulties of processing Big Data. Notably, there are two paradigms of distributed computing that have become increasingly popular over the last decade, namely grid based   
