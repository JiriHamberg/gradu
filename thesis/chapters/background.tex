\section{Background}

Mobile devices are becoming increasingly common~\cite{ITUstats} all around the world. To understand and characterize the different aspects of mobile device usage, various methods of data collection and analysis have been proposed in the literature. Despite technological advances, battery life remains a factor that severely limits the scope of mobile device usage. The literature proposes methods for both accurately modelling energy consumption and for decreasing the energy consumption of devices. Association analysis is a standard data mining technique that has been widely applied. A couple of recent examples of utilizing association analysis for different data mining tasks are shown here to give a sense of the variety of the problems for which the association analysis is applicable. 

\subsection{Mobile Data Collection and Analysis}

Various approaches have been used for collecting and analysing data from mobile devices. Multiple applications have been proposed for collecting detailed logging information from a relatively small number of smart phone users~\cite{Ferreira:2014:CES:2628363.2628367, Falaki:2010:DSU:1814433.1814453, 5375354}. This kind of logging data can be utilized to find device and application usage patterns, to building statistical models that predict individual user's future behaviour, and to building regression models which accurately predict individual users energy consumption. 

Large scale data collection and analysis platforms have also been proposed for studying mobile data. Wagner et al.~\cite{Wagner:2014:DAL:2627534.2627553} describe a mobile data collection system that has been used to collect usage information from 12,500 Android mobile devices. The authors discuss challenges regarding privacy, security, transparency, and accountability of their data collection and analysis system. 

Oliver et al.~\cite{Oliver:2010:CLS:1834616.1834623, Oliver_datadriven} have studied diurnal mobile device usage patterns and energy consumption patterns using a dataset collected from over 15,000 BlackBerry mobile devices. The data consists of the devices backlight activity, the operating system's idle counter measurements, battery level and charging activity, device shut down events, and device type and operating system version.      

Oliner et al.~\cite{Oliner:2012:CED:2387858.2387864, Oliner:2013:CCE:2517351.2517354} have used mobile device system settings and context factors collected from over 500,000 devices to detect energy consumption anomalies in mobile applications. The analysis system called Carat collects samples containing list of currently running applications, level of battery life and other context factors and system settings from the users that have installed the Carat mobile application. These samples are sent to an analysis server which is implemented using Spark cluster-computing framework. The applications with anomalous energy consumption are labelled as energy hogs if they consume above average amount of energy on most users' systems. If an application only consumes more than average energy under specific circumstances, such as on a specific version of an operating system, then the anomaly is labelled as an energy bug. After analysing the samples collected from all users, the analysis server informs the Carat mobile application about discovered energy hogs and bugs. The mobile application can then notify the user about hogs or bugs affecting the user's system and even estimate the amount of battery life that could be saved by shutting off such applications.  

To detect an energy hog application, the authors estimate the reference battery drain probability distribution by using data points that do not have the subject application running. They likewise estimate the subject battery drain probability distribution by using only data points that have the subject application running. If the expected value of the subject distribution is notably higher than the reference distribution's expected value, then the application is labelled as an energy hog. An application that is not an energy hog can be detected as an energy bug by estimating the subject application's energy drain probability distribution assuming some context factors such as a specific operating system version. The energy drain distribution is then estimated by using only data points that have the subject application running and meet the criteria of the context factors. If the subject distribution, given the specific context factors, has an expected value that is notably higher than the reference distribution's expected value, then the application can be labelled as an energy bug. The detected difference in the expected values of the probability distributions gives an estimate of how much energy can be saved by not using an energy hog or bug application. An application can have multiple energy bugs with different sets of context factor criteria. The authors detected a total of 233,258 instances of energy bugs and 10,110 energy hogs from a total of 102,421 applications that were present in the collected dataset.              

\subsection{Battery Life of Mobile Devices}

As mobile devices become increasingly essential for our every day lives, the demand for longer battery lives is constantly increasing. Despite the impact that mobile device battery life has on every day lives of hundreds of millions of people worldwide, the factors which affect a mobile device's battery life have not been studied extensively.

Mobile networking is an active area of research where mobile device energy consumption has an import role. New mobile networking technologies are being developed constantly and providing faster, more reliable and more energy efficient mobile networking solutions for customers is a profitable business for the internet service providers of the world. Consequently, it is no surprise that funding and research efforts have gravitated towards the field.  

As a concrete example of such research, the impact of using 2G and 3G networks for the mobile phone battery life has been compared by Perrucci et al.~\cite{5357972}. The authors used a Nokia N95 phone to test the relative battery consumption of various tasks comparing the results of using GMS, a 2G mobile networking technology and UMTS, a 3G mobile networking technology. The tasks for which the battery consumption was measured included sending 50 SMS messages of 100 bytes, downloading 100 megabytes of data, and performing a 5 hour voice call.
%\begin{itemize}
%\item Sending 50 SMS messages of 100 bytes
%\item Downloading 100 megabytes of data
%\item Performing a 5 hour voice call 
%\end{itemize} 

The conclusion of the study is that different networking technologies are energy efficient in different tasks. While the UMTS network is significantly more energy-efficient for downloading data, the GMS is more energy-efficient when sending SMS messages or performing voice calls. The authors argue that this information could be used to minimize energy consumption of mobile phones when multiple networking technologies are available.       

Another perspective from which the mobile device energy consumption has been studied in recent years, is offloading or remote execution of programs. The idea of offloading is simple: since CPU intensive tasks tend to consume a lot of energy, computing tasks can be executed remotely in a dedicated server or cloud environment. However, transferring data to and from the offloading platform also consumes energy and imposes other constraints such as delay, connectivity, availability, security concerns and potential costs of using such a platform.

One approach to decreasing mobile device energy consumption using computation offloading is proposed by Qian and Andresen~\cite{7176219}. In this work, the authors propose a programming model and a runtime environment called Jade for creating processes that are offloading aware. Programs written using their programming model will be subject to custom task scheduling. The scheduler communicates with available computing servers exchanging information about their available computing resources such as the number of CPUs and the amount of RAM to make decisions on where certain tasks should be offloaded to. Upon first execution of each task that uses Jade, the task will be profiled in order to find out if the program is suitable for offloading in future executions. The profiler collects statistics such as runtime, energy consumption and size of the task. Whenever a remotable task, that has been profiled, is executed, the Jade environment performs an optimization step where it tries to figure out the optimal way to execute the remotable task. The optimizer estimates the energy consumption of the task for each available offloading server as well as the mobile device itself and chooses the host that is estimated to be the most energy efficient for the execution of task.

The authors wrote two applications to evaluate their systems performance. The first program performed facial recognition on photos on the phone, chosen by the user. The other program simulated a navigation application by performing Dijkstra's shortest path finding algorithm on a graph. Both applications were run both locally and with offloading enabled. When using offloading, the programs' energy usage was reduced by 34\% and by 39\% respectively. The execution time of the programs was also reduced by 37\% and 45\% respectively.                 

% begin new content

The usage patterns of mobile devices and applications have also been the subject of various studies and play a role in the energy consumption of the mobile devices. Ferreira et al.~\cite{Ferreira:2014:CES:2628363.2628367} have shown that mobile applications are frequently used in short bursts of activity they call micro-usage. This micro-usage is especially prevalent in social applications and applications that provide users with notifications. The authors suggest that the operating system of a mobile device could possibly optimize resource allocation by identifying applications that are often subject to micro-usage.

Falaki et al.~\cite{Falaki:2010:DSU:1814433.1814453} have studied and characterized the usage patterns of 255 mobile phone users. The authors have discovered a large diversity in the average mobile phone usage statistics such as the amount of network traffic, total energy consumption, number of installed applications, and the diurnal user activity. The authors suggest that energy consumption modelling can be enhanced by incorporating personalized usage statistics into the models. They also demonstrate this idea by implementing a energy drain prediction model that accurately predicts user's energy consumption based on the user's past usage patterns and recent activity.          

Regression models have been used to model energy consumption of Android mobile devices based on context factors and system settings by Shye et al.~\cite{5375354}. The authors show that the CPU usage and screen brightness are the two dominant energy consumption factors in their data which they collect using a custom Android logging application. They also use the data to characterize the users' workloads and develop an application which gradually lowers an Android phone's screen brightness and CPU usage to decrease energy consumption while attempting to keep the changes small enough to be hard for the user to notice. The authors show that the power saving application is able to save up to ten percent of total battery life. For the study, the application was tested by 20 users, 15 of which said they would use the optimizations that the application provides.
% end new content
%The Carat group focuses on studying the impact of various system settings and context factors of a mobile device on its energy usage. The approach is based on crowdsourced measurements collected from the users of Carat mobile application. 

Peltonen et al.~\cite{PELTONEN201671, 7146507} have described a way to model a mobile device's battery life as a function of various context factors and system settings including type, speed and activity of the network that the device was connected to, screen brightness, CPU usage, battery health, voltage and temperature, and the movement of the device. Using conditional mutual information, the model estimates the impact of these factors and variables on the energy consumption of the mobile device. Using this information, the authors construct a decision tree based recommendation system called Constella. The Constella application provides the user of a mobile device with recommended actions to increase remaining battery life. The system compares the current values of the context factors and system variables of the device with the model described above, discovering which changes are estimated by the model to save the most battery life. An example of such recommendation could be "Change from mobile to WiFi network. Expected improvement 33m 33s $\pm$ 57s".    


\subsection{Association Analysis}

Association analysis is a data mining method developed to find common patterns from large databases. The method was famously conceived to find common patterns in shopping cart content databases in order for the supermarkets to optimize the layout of their stores~\cite{Agrawal:1993:MAR:170036.170072}. Since then, the method has been applied to a wide variety of data mining problems.

Karabatak and Cevdet Ince~\cite{KARABATAK20093465} have used association rule generation and a neural network to train an expert system for detecting breast cancer. The input data of the expert system consists of nine variables describing the clump and a cell specimen of the suspected cancer tissue. The authors use the Apriori algorithm~\cite{Agrawal94fastalgorithms} to discover association rules between the input variables. The discovered rules are then used for feature extraction and dimension reduction of the input data. Extracted features are then used as the training data for a multilayer perceptron which is used to classify the tumors to either malignant or benign class. After 3-fold cross validation on a database of 699 records, one combination of association rule assisted feature extraction and neural network achieved a correct classification rate of 97.4\% whereas a network which used the original nine variables as input only achieved a correct classification rate of 95.2\%.

In another study, Karabatak et al.~\cite{KARABATAK201132} have used association analysis to classify textures. In this study, images of textures are transformed using a wavelet transformation. Each pixel is then mapped to range 0 to 2 based on brightness of the transformed pixels. Items for the association rule generation are then generated using a 3 x 3 sliding window, essentially concatenating the nine pixels in the sliding window. Apriori algorithm~\cite{Agrawal94fastalgorithms} is used to generate frequent item sets from the items. The frequent item sets and their related support values are then used to characterize the texture from which the transaction database is generated. In order to classify an unseen texture image, the authors generate the frequent item sets and related support values and use a shortest distance classifier to label the new image to one of the texture classes. Their training and testing data consists of 500 texture images of size 128 x 128. The images are gray scale representations of 10 classes of textures, such as bark, plastic bubbles and brick wall. In the test scenario, the classifier had a success rate of 97\%. 