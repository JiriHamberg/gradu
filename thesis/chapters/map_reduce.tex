\section{Spark and the MapReduce Programming Model}

MapReduce is a programming model and an associated implementation that emerged to simplify common tasks associated with big data processing. These include managing parallel and distributed computing and ensuring fault tolerance of the computations \cite{Dean:2008:MSD:1327452.1327492}. The model is heavily influenced by functional programming, a programming paradigm that emphasises the use of pure functions and avoiding mutable state. As the name \textit{MapReduce} suggests, the computational model of a MapReduce system is based on two higher order functions, \textit{map} and \textit{reduce}. 

Spark is a MapReduce system implemented in the Scala programming language that is built around an abstraction called Distributed Resilient Datasets (RDDs)~\cite{Zaharia:2012:RDD:2228298.2228301}. The RDD abstration allows Spark to implement efficient fault tolerance. A common way of achieving fault tolerance is by maintaining multiple redundant copies of all datasets. Each time a dataset is mutated, all copies are mutated as well. While this certainly makes the system tolerant to lost datasets, it imposes quite significant overhead as each update needs to be replicated and extra space is required by the copies. Instead of maintaining redundant copies of each dataset, Spark solves the problem of fault tolerance by maintains the lineage of its datasets. The lineage of an RDD is a collection of instructions that specify how the RDD was computed from other RDDs. In case the Spark system loses an RDD, it can recreate the lost dataset by tracing its lineage back to existing RDDs and applying the instructions to recompute the lost dataset.       

The programming interface that Spark provides is analogous to the collection library in the Scala standard library. The interface offers two types of functions: transformations that construct new RDDs from existing ones, such as \textit{map} and \textit{filter}, and actions that either save data to disk or return values to the application, such as \textit{collect} and \textit{reduce}~\cite{Zaharia:2012:RDD:2228298.2228301}. Transformations are computed lazily, which allows pipelining consecutive transformations and constructing a lineage graph of the computation before computation even takes place. Actions are used to execute work flows constructed by transformations and return the results to the application or write them to the disk. The map function, defined for class \textit{RDD[T]}, where \textit{T} is the element type parameter of the RDD, has essentially the following type signature

\[map[U](f: (T) \Rightarrow U): RDD[U]\]

The function produces a new RDD with element type \textit{U} by applying function \textit{f} to each element of the original RDD. Similarly, function \textit{filter} has the type signature

\[filter(f: (T) \Rightarrow Boolean): RDD[T]\]

and works by constructing a new RDD by including those elements of the original RDD where predicate \textit{f} is true. The action \textit{collect} has the simple type signature 

\[collect(): Array[T]\]

as it merely returns the results of the computation performed by transformations on the RDD to the application. The \textit{reduce} action has the signature

\[reduce(f: (T, T) \Rightarrow T): T\]

and it works conceptually by applying the binary operator \textit{f} iteratively to elements of the RDD until there is only one element left. The order and pairing of the elements is not specified to allow parallel computation and thus the operator needs to be commutative and associative in order to guarantee deterministic results. 


